\documentclass[a4paper,11pt]{article}
\usepackage[polish]{babel}
\usepackage[utf8]{inputenc}   % lub utf8
\usepackage[T1]{fontenc}
\usepackage{graphicx}
\usepackage{anysize}
\usepackage{enumerate}
\usepackage{times}
\usepackage{tikz}
 
%\marginsize{left}{right}{top}{bottom}
\marginsize{3cm}{3cm}{3cm}{3cm}
\sloppy
 
\begin{document}
 \begin{figure}[!htb]
\centering
\begin{tikzpicture}
\node at (0,0) {\includegraphics[scale=1]{f1.pdf}};
\node at (5.6,-2) {$f(x) = \sin(x)$};
\node at (5.6,1.3) {$g(x) = \sin^2(x)$};
\node at (1.7,0) {$h(x) = \frac{\sin(x)}{2}$};
\end{tikzpicture}
\caption{Wykresy funkcji}
\label{fig:funkcje}
\end{figure}
 
\end{document}