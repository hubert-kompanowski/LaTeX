\documentclass[a4paper,12pt]{book}      
\usepackage[utf8]{inputenc} 
\usepackage[T1]{fontenc}
\usepackage{times}
\usepackage{amssymb}
\usepackage{amsthm}
\usepackage[polish]{babel}

\sloppy


\begin{document}     

\section*{Zastosowania geometryczne całki oznaczonej} 


\paragraph{Obliczanie długości łuku:}
Jeżeli krzywa wyznaczona jest równaniem $y = f(x)$, przy czym \mbox{funkcja}
$...$ ma w przedziale $f(x)$ ciągłą pochodną, to {\em długość łuku} w tym
przedziale wyraża się wzorem: $$L = \int_{a}^{b} \sqrt{1+{[f'(x)]}^2} dx$$

\paragraph{Obliczanie objętości bryły obrotowej:}
Jeżeli $f(x)$ jest funkcją ciągłą i nieujemną na na przedziale $[a,b]$, to
objętość bryły obrotowej powstałej z obrotu wokół osi $Ox$ linii o równaniu
$y=f(x)$, gdzie $x \in [a,b]$, wyraża się wzorem:
$$V = \pi \int_{a}^{b} f^{2}(x) dx$$

\paragraph{Obliczanie pola powierzchni bryły obrotowej:}
Jeżeli $f(x)$ jest funkcją ciągłą i nieujemną na przedziale $[a,b]$ i ma w
tym przedziale ciągłą pochodną, to pole powierzchni bryły obrotowej powstałej z
obrotu wokół osi $Ox$ linii o równaniu  $y=f(x)$, gdzie $x\in[a,b]$, wyraża się
wzorem:
$$S = 2\pi \int_{a}^{b}f(x) \sqrt{1+{[f'(x)]}^2} dx $$

\end{document}
