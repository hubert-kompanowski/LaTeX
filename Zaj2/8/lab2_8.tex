\documentclass[a4paper,12pt]{article}      
\usepackage[utf8]{inputenc} 
\usepackage[T1]{fontenc}
\usepackage{times}
\usepackage{amssymb}
\usepackage{amsthm}
\usepackage[polish]{babel}
\newtheorem*{theorem}{Twierdzenie}
\usepackage{enumerate}
\usepackage{enumitem}

\sloppy

\begin{document}

\begin{theorem}
Jeżeli $\lim \limits_{n \to \infty}  a_{n} = a$ i $ \lim \limits_{n \to \infty}  b_{n} = +\infty$, to:
\begin{enumerate}
\item[$1^\circ$] $\lim \limits_{n \to \infty} (a_{n} + b_{n}) = +\infty$
\item[$2^\circ$] $\lim \limits_{n \to \infty} (a_{n} \cdot b_{n}) = 
\left\{ \begin{array}{111} 
+\infty,& \textnormal{gdy}& a>0,\
-\infty,& \textnormal{gdy}& a<0, 
\end{array} \right.$
\item[$3^\circ$] $\lim \limits_{n \to \infty} \frac{a_{n}}{b_{n}} = 0, $ przy $b_{n}\neq 0$ dla $n \in \mathbb{N} $
\item[$4^\circ$] $\lim \limits_{n \to \infty} \frac{b_{n}}{a_{n}} = 
\left\{ \begin{array}{111} 
+\infty,& \textnormal{gdy}& a>0,\\
-\infty,& \textnormal{gdy}& a<0, 
\end{array} \right.$ przy założeniu że $a_{n} \neq 0 $ dla $n \in \mathbb{N}$.
\end{enumerate}
\end{theorem}
\end{document}