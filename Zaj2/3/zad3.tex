\documentclass[a4paper,12pt]{article}
\usepackage[T1]{fontenc}
\usepackage[latin2]{inputenc}
\usepackage[polish]{babel}
\usepackage{amssymb}
\usepackage{amsthm}
\usepackage{times}
\usepackage{anysize}

\marginsize{1.5cm}{1.5cm}{1.5cm}{1.5cm}
\sloppy 


\begin{document}

Istnieje scisly zwiazek miedzy rozkladem macierzy $\mathbi{A}$ na macierze $\mathbi{L}$ i $\mathbi{U}$ a metoda eliminacji Gaussa. Mozna wykazac, ze elementy kolejnych kolumn macierzy $\mathbi{L}$ sa rowne wspolczynnikom przez ktore mnozone sa w kolejnych krokach wiersze ukladow rownan celem dokonainia eliminacji niewiadomych w odpowiednich kolumnach. Natomiast macierz $\mathbi{U}$ jest rowna macierzy trojkatnej uzyskanej z elimincaji Gaussa.

$$ [A|b] = \left[\begin{array}{rrrr} 
2 & 2 & 4 & 4 \\
1 & 2 & 2 & 4 \\
1 & 4 & 1 & 1 
\end{array}\right] 
= 
\left[\begin{array}{rrrr}
2 & 2 & 4 & 4 \\
0 & 1 & 0 & 2 \\
0 & 3 & -1 & -1 
\end{array}\right]
=
\left[\begin{array}{rrrr}
2 & 2 & 4 & 4 \\
0 & 1 & 0 & 2 \\
0 & 0 & -1 & -7 
\end{array}\right]$$

$$ L = \left[\begin{array}{rrr} 
1 & 0 & 0 \\
\frac{1}{2} & 1 & 0 \\
\frac{1}{2} & 3 & 1 
\end{array}\right]
& \qquad
U = \left[\begin{array}{rrrr} 
2 & 2 & 4 & 4\\
0 & 1 & 0 & 2\\
0 & 0 & -1 & -7
\end{array}\right]
$$


\end{document}
