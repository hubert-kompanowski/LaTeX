\documentclass[a4paper,11pt,fleqn]{article}
\usepackage[utf8]{inputenc}   % lub utf8
\usepackage[T1]{fontenc}
\usepackage{graphicx}
\usepackage{anysize}
\usepackage{enumerate}
\usepackage{hyperref}
\usepackage{xcolor}
\usepackage{tcolorbox}
\usepackage{times}
\usepackage{amsfonts}
\usepackage{amssymb}
\usepackage{amsthm}
\usepackage{amsmath}
\usepackage{graphicx}
\usepackage{sidecap}
\usepackage{wrapfig}
\usepackage[polish, british]{babel}
\usepackage{caption}
\usepackage[export]{adjustbox}
%\marginsize{left}{right}{top}{bottom}
\marginsize{3.5cm}{2.5cm}{3cm}{2cm}
%\newtheorem{theorem}{Theorem }
\begin{document}
\selectlanguage{polish}
\setcounter{section}{5}
\setcounter{page}{123}
 
 
\section{Instrukcja}
Jest to ,,sztuczny dokument`` klasy \textit{artykuł} złozony z szóstej sekcji (Instrukcji), siódmej sekcji
(w języku angielskim) oraz spisu literatury. Papier A4, podstawowy rozmiar czcionki 11pt, margines
górny ma 3cm, dolny 2 cm, lewy 3,5 cm, prawy 2,5 cm.
 
 
 
\subsection{Co należy zrobić}\label{conalezyzrobic}
\begin{enumerate}
\item Nalezy utworzyć w {\LaTeX}u dokument jak najblizszy temu, który Pani/Pan teraz czyta\footnote{W kolejnych podsekcjach są wskazania na szczegóły, na które warto zwrócic uwagę.}.
\item Na początku pliku źródłowego (w komentarzu) proszę umieścić swoje Imię i Nazwisko.
\item Plik źródłowy (\textit{.tex}) i ew. inne pliki wymagane przy kompilacji oraz plik wynikowy (\textit{.pdf})
należy wysłać na adres \href{mailto:miller@agh.edu.pl}{\nolinkurl{miller@agh.edu.pl} } jako załączniki wiadomości o tytule zawierającym Nazwisko i Imię. Jeżeli kompresja byłaby konieczna, to proszę zastosować
format \textit{zip}.
\end{enumerate}
 
 
 
\subsection{Na co zwrócić uwagę}
Proszę zwrócic uwagę na \colorbox{red}{\textbf{rozmiar papieru, czcionki i marginesów}} oraz na wymienione nizej szczegóły\footnote{Odstępstwa od wzorca w tych miejscach nie dyskwalifikują pracy, ale mogą spowodować delikatne obniżenie oceny}.
\begin{itemize}
\item Na końcach linii nie ma wyrazów jednoliterowych.
\item Wyrażenia (\ref{equation1a}–\ref{equation1c}) są wyrównane w pionie wg znaku nierówności.
\item Numeracja tych nierównosci jest złożona z liczby i litery – w wersji uproszczonej może
zawierać kolejne liczby.
\item Odwołania do wzorów, sekcji, literatury itp. powinny być „elastyczne” — należy nadawać
odpowiednie etykiety i skorzystać z nich przy odwołaniu.
\item Numeracja sekcji, twierdzeń i stron nie zaczyna się od 1.
\item Pisząc twierdzenie należy użyć odpowiedniego stylu.
\item Dowód jest zakonczony specjalnym znakiem.
\item Odwołanie do literatury można ograniczyć do numeru pozycji — bez numeru rozdziału, paragrafu i twierdzenia.
\item[*] Tworząc spis literatury można (bez obniżenia oceny) zamiast Bib{\TeX}a użyć środowiska
\textcolor{blue}{thebibliography}.
\item[*] Mój adres mailowy w (\ref{conalezyzrobic}) jest napisany z uzyciem pakietu \textit{hyperref}, ale mozna tę postać osiągnąć ,,ręcznie`` — bez obniżenia oceny
\item W tym wypunktowaniu używany jest inny symbol (gwiazdka) w przypadkach, gdy odejście od wzorca nie obniza oceny.
\end{itemize}
 
\newpage\selectlanguage{british}
\section{Differential Inequalities}
\theoremstyle{definition}
\newtheorem{theorem}{Theorem}
\setcounter{theorem}{7}
\begin {theorem}
\small {Suppose that the functions $m(x)$ and $u(x)$are continuous and satisfy for $x_0 \leq x < X$}
\end{theorem}
 
\begin{subequations}
\begin{align}
&&D + m(x) \leq& \:,g (x, m(x))\label{equation1a}\\
&&D + u(x) >& \:g(x, u(x))\label{equation1b}\\
&&m(x0) \leq& \:u(x0)\label{equation1c}.
\end{align}
\end{subequations}
 
Then
 
\begin{align}\label{equation2}
&&m(x) \leq u(x)\quad &\text{for}\quad x_0 \leq x \leq X.
\end{align}
 
The same conclusion is true if both $D_+$ are replaced by $D^+$.
\begin{proof}
If (\ref{equation2}) were not true, we could choose a point $x_2$ with $m(x_2) > u(x_2)$ and look for the first
point $x_1$ to the left of $x_2$ with $m(x_1) = u(x_1)$ . Then for small $h > 0$ we would have
$$\frac{m(x_1 + h) - m(x_1)}{h} > \frac{u(x_1 + h) - u(x_1)}{h}$$
and, by taking limits, $D_+m(x_1) \leq D_+u(x_1)$. This, however, contradicts (\ref{equation1a}) and (\ref{equation1b}), which give
$$D_+m(x_1) \leq g(x_1, m(x_1)) = g(x_1, u(x_1)) < D_+u(x_1).$$
\end{proof}
Many variant forms of this theorem are possible, for example by using left Dini derivates~\cite[Chap. II, §8, Theorem V]{Walter}.
 
\begin{thebibliography}{1}
\bibitem{Walter}
W. Walter (1970): Differential and integral inequalities. Springer Verlag 352pp., german edition 1964.
\end{thebibliography}
 
\end{document}