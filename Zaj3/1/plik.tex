\documentclass[a4paper,11pt, fleqn]{article}

\usepackage[cp1250]{inputenc}   % lub utf8
\usepackage[T1]{fontenc}
\usepackage{graphicx}
\usepackage{anysize}
\usepackage{enumerate}
\usepackage{times}
\usepackage{amssymb}
\usepackage{amsmath}
\usepackage[polish]{babel}
\usepackage{wrapfig}
\usepackage{caption}


%\marginsize{left}{right}{top}{bottom}
\marginsize{3cm}{3cm}{3cm}{3cm}
\sloppy
 
\begin{document}
\bibliographystyle{abbrv}
 
Z ka�dym dzia�aj�cym systemem komputerowym powi�zane jest oczekiwanie 
{\em poprawno�ci} jego dzia�ania \cite{sommerville2011software}. Istnieje szeroka 
klasa system�w, dla kt�rych poprawno�� powi�zana jest nie tylko z 
wynikami ich pracy, ale r�wnie� z~czasem, w~jakim wyniki te s� 
otrzymywane. Systemy takie nazywane s� {\em systemami czasu 
rzeczywistego}, a~poniewa� s� one rozpatrywane  w~kontek�cie swojego 
otoczenia, cz�sto okre�lane s� terminem {\em systemy wbudowane} 
\cite{sommerville2011software}\cite{metody:formalne}. 

Ze wzgl�du na specyficzne cechy takich system�w, weryfikacja jako�ci 
tworzonego oprogramowania oparta wy��cznie na jego testach jest 
niewystarczaj�ca. Coraz cz�ciej w~takich sytuacjach, weryfikacja 
poprawno�ci tworzonego systemu lub najbardziej istotnych jego 
modu��w prowadzona jest z~zastosowaniem metod formalnych 
\cite{Alur+Dill/90/Automata}\cite{metody:formalne}.
\bibliography{bibliografia}
 
 
\end{document}